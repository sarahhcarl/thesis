\abstract{
Drug repositioning is the discovery of new indications for approved or failed drugs. This practice is commonly done within the drug discovery process in order to adjust or expand the application line of an active molecule. Nowadays, an increasing number of computational methodologies aim at predicting repositioning opportunities in an automated fashion. Some approaches rely on the direct physical interaction between molecules and protein targets (docking) and some methods consider more abstract descriptors, such as a gene expression signature, in order to characterise the potential pharmacological action of a drug (Chapter 1).

On a fundamental level, repositioning opportunities exist because drugs perturb multiple biological entities, (on and off-targets) themselves involved in multiple biological processes. Therefore, a drug can play multiple roles or exhibit various mode of actions responsible for its pharmacology. The work done for my thesis aims at characterising these various modes and mechanisms of action for approved drugs, using a mathematical framework called description logics.

In this regard, I first specify how living organisms can be compared to complex black box machines and how this analogy can help to capture biomedical knowledge using description logics (Chapter 2). Secondly, the theory is implemented in the Functional Therapeutic Chemical Classification System (FTC - https://www.ebi.ac.uk/chembl/ftc/), a resource defining over 20,000 new categories representing the modes and mechanisms of action of approved drugs. The FTC also indexes over 1,000 approved drugs, which have been classified into the mode of action categories using automated reasoning. The FTC is evaluated against a gold standard, the Anatomical Therapeutic Chemical Classification System (ATC), in order to characterise its quality and content (Chapter 3).

Finally, from the information available in the FTC, a series of drug repositioning hypotheses were generated and made publicly available via a web application (https://www.ebi.ac.uk/chembl/research/ftc-hypotheses). A subset of the hypotheses related to the cardiovascular hypertension as well as for Alzheimer’s disease are further discussed in more details, as an example of an application (Chapter 4).

The work performed illustrates how new valuable biomedical knowledge can be automatically generated by integrating and leveraging the content of publicly available resources using description logics and automated reasoning. The newly created classification (FTC) is a first attempt to formally and systematically characterise the function or role of approved drugs using the concept of mode of action. The open hypotheses derived from the resource are available to the community to analyse and design further experiments.
}
