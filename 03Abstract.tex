\abstract{
Genome-wide binding and expression studies in \emph{Drosophila melanogaster} have revealed widespread roles for Dichaete and SoxNeuro, two group B Sox proteins, during fly development. Although they have distinct target genes, these two transcription factors bind in very similar patterns across the genome and can partially compensate for each other's loss, both phenotypically and at the level of DNA binding. However, the inherent noise in genome-wide binding studies as well as the high affinity of transcription factors for DNA and the potential for non-specific binding makes it difficult to identify true functional binding events. Additionally, external factors such as chromatin accessibility are known to play a role in determining binding patterns in \emph{Drosophila}. A comparative approach to transcription factor binding facilitates the use of evolutionary conservation to identify functional features of binding patterns. In order to discover highly conserved features of group B Sox binding, I performed DamID-seq for SoxNeuro and Dichaete in four species of \emph{Drosophila, D. melanogaster, D. simulans, D. yakuba} and \emph{D. pseudoobscura}. I also performed FAIRE-seq in \emph{D. pseudoobscura} embryos to compare the chromatin accessibility landscape between two fly species and to examine the relationship between open chromatin and group B Sox binding. 

I found that, although the sequences, expression patterns and overall transcriptional regulatory targets of Dichaete and SoxNeuro are highly conserved across the drosophilids, both binding site turnover and rates of quantitative binding divergence between species increase with phylogenetic distance. Elevated rates of binding conservation can be found at bound genomic intervals overlapping functional sites, including known enhancers, direct targets of Dichaete and SoxNeuro, and core binding intervals identified in previous genome-wide studies. Sox motifs identified in intervals that show binding conservation are also more highly conserved than those in intervals that are only bound in one species. Notably, regions that are bound in common by SoxNeuro and Dichaete are more likely to be conserved between species than those bound by one protein alone. However, by examining binding intervals that are uniquely bound by one protein and conserved, I was able to identify distinctive features of the targets of each transcription factor that point to unique aspects of their functions. 

My comparative analysis of group B Sox binding suggests that sites that are commonly bound by Dichaete and SoxNeuro, primarily at targets in the developing nervous system, are highly constrained by natural selection. Uniquely bound targets have different tissue expression profiles, leading me to propose a model whereby the unique functions of Dichaete and SoxNeuro may arise from a combination of differences in their own expression patterns and the broader nuclear environment, including tissue-specific cofactors and patterns of accessible chromatin. These results shed light on the evolutionary forces that have maintained deep conservation of the complex functional relationships between group B Sox proteins from insects to mammals.
}
