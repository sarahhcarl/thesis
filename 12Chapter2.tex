\chapter{Materials and Methods}

\hrulefill

\section{Fly husbandry and stock keeping}
The wild-type strains of the following \emph{Drosophila} species were used in all experiments: \emph{D. melanogaster} Oregon-R and \emph{w\textsuperscript{1118}}; \emph{D. simulans w\textsuperscript{[501]}} (reference strain - \url{http://www.ncbi.nlm.nih.gov/genome/200?genome_assembly_id=28534}); \emph{D. yakuba} Cam-115 \citep{coyne_genetic_2004}; \emph{D. pseudoobscura pseudoobscura} (reference strain - \url{http://www.ncbi.nlm.nih.gov/genome/219?genome_assembly_id=28567}). \emph{D. melanogaster, D. simulans} and \emph{D. yakuba} flies were kept at 25\degree~C on standard cornmeal medium. \emph{D. pseudoobscura} flies were kept at 22.5\degree~C in low humidity, on banana-opuntia-malt medium (1000 ml water, 30 g yeast, 10 g agar, 20 ml Nipagin, 150 g mashed banana, 50 g molasses, 30 g malt, 2.5 g opuntia powder). All embryo collections were performed at 25\degree~C with the exception of the \emph{D. pseudoobscura} staged collections for FAIRE-seq, which were performed at 22.5\degree~C. Flies were allowed to lay for varying periods of time on agar plates supplemented with grape juice and streaked with fresh yeast paste.\\

All microinjections to generate transgenic lines were performed by Sang Chan in the Department of Genetics injection facility. Before injections, flies were kept in cages for 2 days at 25\degree~C, with a fresh grape juice-agar plate with yeast paste provided twice a day. After 2 days, the plates were changed every 30 minutes for 2 hours, and then embryos were collected after a 30-minute lay. In an 18\degree~C injection room, embryos were washed and dechorionated in 50\% bleach for 3 minutes. They were then rinsed with cold water, blotted dry on a paper towel and transferred with a paintbrush to a coverslip on which a stripe of heptane-glue had been painted (made by dissolving sellotape in heptane). The embryos were aligned on the heptane-glue with forceps and covered with 10 S Voltalef oil (VWR). The posterior end of each embryo was injected using a glass needle loaded on a Leitz micromanipulator. The injection mix consisted of a piggyBac helper plasmid at 0.4 \(\mu\)g/\(\mu\)l and a piggyBac plasmid containing the construct of interest at 0.6 \(\mu\)g/\(\mu\)l.\\

Injected embryos were transferred on the coverslip to a grape juice-agar plate with a small dot of yeast paste and left to develop for 24 hours at 25\degree~C. \emph{D. pseudoobscura} embryos were allowed to develop for up to 48 hours to account for slower developmental times. Any hatched larvae were then transferred with the yeast paste into a fresh tube containing cornmeal medium. For \emph{D. simulans, D. yakuba} and \emph{D. pseudoobscura}, surviving adults were backcrossed to males or virgin females from the parental, wild-type strain. F1 progeny were then scored for eye-specific GFP expression, and transgenic lines were set up by crossing GFP-positive siblings. Because I was unable to identify flies carrying two copies of the transgene, these lines consisted of a mixed population of homozygous and heterozygous flies, meaning that the populations had to be periodically checked and GFP-positive flies selected in order to prevent loss of the transgene through genetic drift. For \emph{D. melanogaster}, surviving adults were backcrossed to \emph{w; Sco/SM6a} males or virgin females. F1 males were scored for eye-specific GFP expression and crossed singly to \emph{w; Sco/SM6a} virgins, then the same males were crossed to \emph{w; TM2/TM6c} virgins. F2 progeny of the \emph{Sco/SM6a} cross were scored for eye-specific GFP expression and a curly wing phenotype, while F2 progeny of the \emph{TM2/TM6c} cross were scored for eye-specific GFP expression and a Stubble phenotype. Siblings of each class were mated together. Balanced transgenic lines were identified in the F3 generation as stocks where all flies showed eye-specific GFP expression. 

\section{Immunohistochemistry}
Embryos were collected from each species after an overnight lay following the protocol described above. They were then dechorionated in 50\% bleach for 3 minutes, rinsed in cold water, and fixed by shaking for 20 minutes in 1.8 ml fixation solution (0.1 M PIPES, 1 mM MgSO4, 2 mM EGTA, pH 6.9) with 0.5 ml formaldehyde and 4 ml heptane. The aqueous phase was removed and 6 ml of methanol was added, followed by vortexing for 30 seconds. Any embryos that sank to the bottom of the tube were collected, rinsed with methanol, and stored at -20\degree~C until needed for staining. Staining was performed as described \citep{patel_imaging_1994} with primary antibodies at the following concentrations: rabbit anti-Dichaete, 1:100; rabbit anti-SoxN, 1:100 or 1:50. Primary antibodies were detected with biotin-conjugated secondary antibodies (goat anti-rabbit) at 1:200 using the ABC Elite kit (Vectastain). Stained embryos were mounted in 70\% glycerol and photographed using Openlab v.4.0.2 imaging software on a Zeiss Axioplan microscope with a 20x objective.

\section{Chromatin immunoprecipitation}
For chromatin immunoprecipitations (ChIP), embryos were collected after an overnight or 12-hour lay and dechorionated as described above. They were fixed by shaking for 20 minutes in 670 \(\mu\)l crosslinking solution (50 mM HEPES, 1mM EDTA, 0.5 mM EGTA, 100 mM NaCl, pH 8.0) with 33 \(\mu\)l 37\% formaldehyde and 3 ml heptane added. The crosslinking reaction was stopped by centrifuging for 2 minutes at 1000g to pellet the embryos, removing the supernatant and adding 2 ml PBT with 125 mM glycine. Embryos were then weighed in an Eppendorf tube, flash-frozen in liquid nitrogen and stored at -80\degree~C. Approximately 200 mg of embryos were used per biological replicate. ChIPs were performed as described with some modifications for a small amount of starting material \citep{ghavi-helm_analyzing_2012, sandmann_chip--chip_2007}. Embryos were homogenized in Eppendorf tubes using a plastic pestle rather than in a Dounce homogenizer. Each sample was homogenized for 30 seconds in 1 ml cold PBT supplemented with protease inhibitors (Complete Mini Protease Inhibitor cocktail tablets, Roche), then allowed to rest on ice for 30 seconds, then homogenized again for 30 seconds. The lysate was spun at 400g for 1 minute at 4\degree~C, and the supernatant decanted into a fresh Eppendorf tube. After centrifugation at 1100g for 10 minutes at 4\degree~C, the supernatant was discarded and the pellet resuspended in 1 ml cold cell lysis buffer supplemented with protease inhibitors. The sample was homogenized again for 30 seconds with a plastic pestle and the lysate spun at 2000g for 4 minutes at 4\degree~C to pellet the nuclei. The pellet was resuspended in 1 ml cold nuclear lysis buffer and incubated for 20 minutes at room temperature to lyse the nuclei.\\ 

A Diagenode Bioruptor was used for sonication, with the energy settings on high. Chromatin was sonicated in 100 \(\mu\)l aliquots for 16 cycles of 30 seconds on, 30 seconds off. A 50 \(\mu\)l input aliquot was removed from each sample and treated with RNaseA for 30 minutes at 37\degree~C, then with proteinase K overnight at 37\degree~C. Crosslinks were reversed by incubating at 65\degree~C for 6 hours. The distribution of DNA fragment sizes was assessed by performing a phenol-chloroform extraction and running the resulting DNA on a 3\% agarose gel. Fragment sizes ranged from approximately 100 bp to 1000 bp, with the majority of the fragments falling between 300 and 700 bp. Immunoprecipitation was carried out with protein A-agarose beads (Millipore), using the buffers and wash protocol described in Sandmann et al. (2006). Anti-Dichaete antibody was pre-cleared by incubating for 3 hours at 4\degree~C with methanol-fixed embryos, then added to a final concentration of 1:300. Affinity-purified anti-SoxN antibody was added without pre-clearing to a final concentration of 1:100. For mock IP controls, a rabbit anti-beta galactosidase antibody (AbCam) was added to a final concentration of 1:1000. After performing the immunoprecipitation, crosslinks were reversed and the DNA purified using the same protocol described above for input samples.

\subsection{ChIP-PCR}
Targets were chosen for PCR amplification to test the specific enrichment of each ChIP by examining previous ChIP-chip and DamID experiments carried out in our lab \citep{aleksic_role_2013,ferrero_soxneuro_2014}. For each gene, a highly bound interval was identified in \emph{D. melanogaster} and its sequence was used as a query to search the genome of each other species using BlastN \citep{altschul_basic_1990}, with the goal of identifying an orthologous region of 500-800 bp. A negative control region was also identified for each factor in each species where binding was not observed in previous experiments in \emph{D. melanogaster}. Primers were designed to amplify each region using Primer3 Plus \citep{untergasser_primer3plus_2007}. Oligonucleotide sequences are shown in Table 2.1. PCR conditions were identical for each set of samples and were as follows: 95\degree~C for 2 min.; 45 cycles of 95\degree~C for 30 sec., 58\degree~C for 30 sec., 72\degree~C for 30 sec.; 72\degree~C for 5 min. 1 \(\mu\)l of ChIP, mock IP, or input DNA was used as a template for each reaction. PCR products were run out on a 1\% agarose gel, and the specificity of each antibody was assayed by comparing the presence and brightness of bands for the ChIP samples versus the mock IP and input samples.

\begin{center}
\begin{longtable}{|l|l|p{2.2cm}|p{3.3cm}|p{3.2cm}|}
\hline
\textbf{TF}       & \textbf{Target gene} & \textbf{Species}          & \textbf{Forward primer (5'-3')}       & \textbf{Reverse primer (5'-3')}       \\ \hline
\endfirsthead

\caption{Primers used to amplify target sequences of Dichaete and SoxN in each species for ChIP-PCR.}
\endlastfoot

Dichaete & \emph{slit}           & \emph{D. melanogaster}  & \parbox[t]{3cm}{GATGCGAACC\\ CAACTGAACT} & \parbox[t]{3cm}{AAACTCAAAC\\ GTGCCGTAGA} \\ \hline
         & \emph{achaete}        & \emph{D. melanogaster}  & \parbox[t]{3cm}{TGATGTCTGG\\ ACCTTGTTGC} & \parbox[t]{3cm}{CCATTAAAGG\\ CCGAAGATGA} \\ \hline
         & \emph{comm}           & \emph{D. melanogaster}  & \parbox[t]{3cm}{AGAACCGGTT\\ TTCGAGTGG}  & \parbox[t]{3cm}{ATAAGCCTGA\\ GCGCGAAGTT} \\ \hline
         & \emph{klingon} (neg.) & \emph{D. melanogaster}  & \parbox[t]{3cm}{ATCCGAATTC\\ AAATCCACCA} & \parbox[t]{3cm}{GCAATCGAAA\\ AAGTGGCAAT} \\ \hline
         & \emph{slit}           & \emph{D. simulans}      & \parbox[t]{3cm}{GATGCGAACC\\ CAACTGAACT} & \parbox[t]{3cm}{GCCACAGACA\\ ATGCGACTTA} \\ \hline
         & \emph{achaete}        & \emph{D. simulans}      & \parbox[t]{3cm}{TGATGTCTGG\\ ACCTTGTTGC} & \parbox[t]{3cm}{TTAACGGCCG\\ AAGATGATTC} \\ \hline
         & \emph{comm}           & \emph{D. simulans}      & \parbox[t]{3cm}{GAACGCAAAA\\ TCTCGACCAT} & \parbox[t]{3cm}{AGTGACATTC\\ CATGGGGAGA} \\ \hline
         & \emph{klingon} (neg.) & \emph{D. simulans}      & \parbox[t]{3cm}{CAAAATCAGG\\ AGCAGCACAA} & \parbox[t]{3cm}{GGATGTTGGA\\ TTTGGATTCG} \\ \hline
         & \emph{slit}           & \emph{D. yakuba}        & \parbox[t]{3cm}{AGTGACATTC\\ CATGGGGAGA} & \parbox[t]{3cm}{ATACGTGCCA\\ CAGACAATGC} \\ \hline
         & \emph{achaete}        & \emph{D. yakuba}        & \parbox[t]{3cm}{ATACAAATTG\\ CATGGCCACA} & \parbox[t]{3cm}{GAGACGATGG\\ TCCTTGCTTC} \\ \hline
         & \emph{comm}           & \emph{D. yakuba}        & \parbox[t]{3cm}{AGGGAAATGG\\ GAAAATCCAC} & \parbox[t]{3cm}{AAAGTGGCCA\\ AGAGCTGAAA} \\ \hline
         & \emph{klingon} (neg.) & \emph{D. yakuba}        & \parbox[t]{3cm}{CAAAATCAGG\\ AGCAGCACAA} & \parbox[t]{3cm}{GAATGTTGCA\\ TTTGCCTCCT} \\ \hline
         & \emph{slit}           & \emph{D. pseudoobscura} & \parbox[t]{3cm}{GCTGTGGACA\\ CACACTCACC} & \parbox[t]{3cm}{GCGAGACCCG\\ TAAAACAGTC} \\ \hline
         & \emph{achaete}        & \emph{D. pseudoobscura} & \parbox[t]{3cm}{CCACCCCTGA\\ TTTATTGTGG} & \parbox[t]{3cm}{CAGCATCAAT\\ GTGGCTCACT} \\ \hline
         & \emph{comm}           & \emph{D. pseudoobscura} & \parbox[t]{3cm}{CTCTCGGGCT\\ GTACTCAAGG} & \parbox[t]{3cm}{TTCCGTTCCT\\ TGTTTGTTCC} \\ \hline
         & \emph{klingon} (neg.) & \emph{D. pseudoobscura} & \parbox[t]{3cm}{ATAGCCACGT\\ AAGCCAATCG} & \parbox[t]{3cm}{GGGGGAGCAA\\ AGTATTAGCC} \\ \hline
SoxN     & \emph{nerfin-1}       & \emph{D. melanogaster}  & \parbox[t]{3cm}{GAGCCCATTG\\ AAAAGCTCAG} & \parbox[t]{3cm}{GCTCGTCGTC\\ ATAGCTCTCC} \\ \hline
         & \emph{gcm-2}          & \emph{D. melanogaster}  & \parbox[t]{3cm}{GCCGTATGTG\\ GAGGACAACT} & \parbox[t]{3cm}{GTGATGGTGA\\ TGGTGGTACG} \\ \hline
         & \emph{castor}         & \emph{D. melanogaster}  & \parbox[t]{3cm}{ACCTCTATCC\\ GGGAATGACC} & \parbox[t]{3cm}{TTGGTTTTTG\\ TGGAGGGAAG} \\ \hline
         & \emph{ppd6} (neg.)    & \emph{D. melanogaster}  & \parbox[t]{3cm}{AATTCGGTGG\\ AAACGATCAC} & \parbox[t]{3cm}{ACCTCGATCA\\ CTCGATGTCC} \\ \hline
         & \emph{nerfin-1}       & \emph{D. simulans}      & \parbox[t]{3cm}{CTGAAAACCA\\ GGTGCGAAAT} & \parbox[t]{3cm}{GAGTGGCTTT\\ ATTGCGGAAG} \\ \hline
         & \emph{gcm-2}          & \emph{D. simulans}      & \parbox[t]{3cm}{GCCGTATGTG\\ GAGGACAACT} & \parbox[t]{3cm}{GGTGGTGATG\\ GTGGTAGGTC} \\ \hline
         & \emph{castor}         & \emph{D. simulans}      & \parbox[t]{3cm}{GCCACCCAAG\\ AAAATCGTAA} & \parbox[t]{3cm}{GGTCATTCCC\\ GGATAGAGGT} \\ \hline
         & \emph{ppd6} (neg.)    & \emph{D. simulans}      & \parbox[t]{3cm}{AACTCGGTGG\\ AAACGATCAC} & \parbox[t]{3cm}{GGTAGCTAAC\\ ACCCCGACA}  \\ \hline
         & \emph{nerfin-1}       & \emph{D. yakuba}        & \parbox[t]{3cm}{CTGAAAACCA\\ GGTGCGAAAT} & \parbox[t]{3cm}{TGGTTTTAGG\\ CGCTGTATCC} \\ \hline
         & \emph{gcm-2}          & \emph{D. yakuba}        & \parbox[t]{3cm}{AACAGTACGG\\ CGGAAATCAG} & \parbox[t]{3cm}{TGAGTAATCC\\ TCCGGTGTCC} \\ \hline
         & \emph{castor}         & \emph{D. yakuba}        & \parbox[t]{3cm}{CTCTTCCAGC\\ TGCAAAATCC} & \parbox[t]{3cm}{TCAAAGTGTG\\ GCTGAGTTGG} \\ \hline
         & \emph{ppd6} (neg.)    & \emph{D. yakuba}        & \parbox[t]{3cm}{AATTCGGTGG\\ AAACGATCAC} & \parbox[t]{3cm}{ACCTCGATCA\\ CTCGATGTCC} \\ \hline
         & \emph{nerfin-1}       & \emph{D. pseudoobscura} & \parbox[t]{3cm}{ACCGCAGTCG\\ CTATCTGAAT} & \parbox[t]{3cm}{TCCTCCTCTT\\ CGTCGATGTT} \\ \hline
         & \emph{gcm-2}         & \emph{D. pseudoobscura}  & \parbox[t]{3cm}{TACGAGTCGA\\ GTCCCCAGTT} & \parbox[t]{3cm}{GCGCTCTCGT\\ AGAAGTGTCC} \\ \hline
         & \emph{castor}         & \emph{D. pseudoobscura} & \parbox[t]{3cm}{CCACCCCTCT\\ CTCCTCTCTC} & \parbox[t]{3cm}{TGGTACAAGA\\ GGGGGTTCTG} \\ \hline
         & \emph{ppd6} (neg.)    & \emph{D. pseudoobscura} & \parbox[t]{3cm}{TGGAGGAGAG\\ CAAGAGGAAA} & \parbox[t]{3cm}{AGTTGACCAA\\ TGGCGGATAG} \\ \hline
\end{longtable}

\label{Table 2.1}
\end{center}

\subsection{ChIP-chip}
Dichaete ChIP samples from \emph{D. melanogaster} were hybridized to a dual-color Nimblegen HD2 (2.1M probe) whole-genome tiling array in order to validate the specificity of the immunoprecipitation reactions. Probe libraries were constructed as described \citep{sandmann_chip--chip_2007}. ChIP samples and their respective mock IP controls were labelled with either Cy3 or Cy5 dyes, with a dye swap in one of three biological replicates. Each ChIP sample and its matched control were hybridized to the same microarray. Hybridization was performed according to the manufacturer’s specifications. Spot-finding was carried out using NimbleScan, a proprietary software package developed by Roche. The raw data were quantile-normalized in R and analyzed with two different peak-calling algorithms, TiMAT (http://bdtnp.lbl.gov/TiMAT/) and Ringo \citep{toedling_ringo_2007} at FDR values of 1\%, 5\%, 10\% and 25\%.

\subsection{ChIP-seq}
ChIP reactions for ChIP-sequencing were performed as described above, with the exception that the protein A-agarose beads were changed to protein A/G PLUS-agarose beads (Santa Cruz Biotechnology), as these do not contain salmon sperm DNA, a potential sequencing contaminant. Before library construction, sample concentrations were measured on a Qubit using the DNA High Sensitivity Assay (Life Technologies). Initially, libraries were constructed for sequencing in-house on an Ion Torrent PGM using the Ion Plus Fragment Library Kit (Life Technologies), quantified via qPCR, and templated using the Ion OneTouch Template Kit (Life Technologies). Libraries were sequenced on 316 chips; however, the quality and coverage of the resulting reads was insufficient for identifying binding peaks.\\

For the first attempt at Illumina sequencing, libraries were prepared using 10 ng of ChIP DNA or mock IP DNA, or the entire sample if less than 10 ng were available, with the NEBNext ChIP-Seq Library Prep Master Mix Set for Illumina (NEB). Samples were barcoded using the NEBNext Multiplex Oligos for Illumina (Index Primers 1-12) (NEB). For the second attempt, libraries were constructed using 10 ng of ChIP or input DNA, or the entire sample if less than 10 ng were available, with the TruSeq DNA LT Sample Prep kit (Illumina). Samples were barcoded using the indexed adapters included in the kit, which were diluted 1:250 to account for the low amount of starting material. Size selection was performed using Agencourt AMPure XP beads (Beckman Coulter), with the aim of recovering fragments between 250 and 400 bp. In all cases, the library concentrations were measured using a Qubit with the DNA High Sensitivity Assay (Life Technologies), and the size distributions of DNA fragments were measured using a 2100 Bioanalyzer with the High Sensitivity DNA kit and chips (Agilent). 10 nM libraries were sent to the EMBL GeneCore sequencing facility in Heidelberg, Germany (http://genecore3.genecore.embl.de/genecore3/index.cfm) for sequencing on an Illumina HiSeq 2000 with v3 chemistry, with 12 samples multiplexed per lane. All libraries were sequenced as 50-bp single-end reads.

\section{DamID}
\subsection{Cloning}
Three constructs were created for DamID: Dichaete-Dam, SoxN-Dam and Dam-only. The SoxN-Dam fusion protein coding sequence was initially cloned from an existing pUAST vectors (from Enrico Ferrero). Primers were designed to amplify the coding regions of the SoxN-Dam fusion protein as well as upstream UAS sites, an \emph{HSP70} promoter, and the \emph{SV40} 5' UTR. An \emph{Spe}I site was introduced upstream of the cloned region. The Dichaete-Dam fusion protein coding sequence was cloned from genomic DNA extracted from a \emph{D. melanogaster} line carrying this construct, which was created by Faysal Riaz \citep{riaz_phd}. Primers were designed to amplify the fusion protein coding region, upstream UAS sites, an \emph{HSP70} promoter, and the \emph{kayak} 5' UTR. A forward primer was used to introduce an \emph{Spe}I site upstream of the cloned region, and a reverse primer introduced an \emph{Avr}II site downstream of the Dichaete-Dam cloned region. The Dam coding region, as well as upstream UAS sites, an \emph{HSP70} promoter and the \emph{SV40} 5' UTR, was cut directly out of an existing pUAST vector (from Tony Southall). All primer sequences are available in Table 2.2. 

\begin{table}[h]
\centering
\begin{tabular}{|l|p{5cm}|p{5cm}|}
\hline
\textbf{Target}       & \textbf{Forward Primer}                  & \textbf{Reverse Primer}                  \\ \hline
Dichaete-Dam & \parbox[t]{5cm}{GGGACTAGTCGAGTAC\\ GCAAAGCTTCTGCAT} & \parbox[t]{4cm}{GGGCCTAGGAGTAAG\\ GTTCCTTCACAAAGAT} \\ \hline
SoxN-Dam     & \parbox[t]{5cm}{GGGACTAGTCGAGTAC\\ GCAAAGCTTCTGCAT} & \parbox[t]{4cm}{GCGCTGACTTTGAGT\\ GGAAT} \\ \hline          
\end{tabular}
\caption{Primers used to amplify Dichaete-Dam and SoxN-Dam coding sequences and flanking regions in \emph{D. melanogaster}.}
\label{Table 2.2}
\end{table}

A two-step cloning process was employed, first cloning inserts into the pSLfa1180fa shuttle vector, and then from the shuttle vector into a \emph{piggyBac} vector marked with 3xP3-EGFP, pBac{3xP3-EGFPafm} \citep{horn_versatile_2000}. To clone all inserts into the shuttle vector, PCR amplicons and the pSLfa1180fa vector were cut with the following restriction enzymes (NEB):
\begin{itemize}
	\item SoxN-Dam, pSLfa1180fa: \emph{Spe}I/\emph{Stu}I
	\item Dichaete-Dam, pSLfa1180fa: \emph{Spe}I/\emph{Avr}II
	\item Dam, pSLfa1180fa: \emph{Sph}I/\emph{Stu}I
\end{itemize}
Digestions were performed at 37\degree~C for 1.5 hours and were stopped by incubation at 65\degree~C for 20 minutes. All samples of cut pSLfa1180fa were dephosphorylated after restriction digestion by incubation with Antarctic phosphatase (NEB) at 37\degree~C for 1 hour. The dephosphorylation reaction was stopped by incubation at 70\degree~C for 10 minutes. Digestion products were run out on a 0.8\% agarose gel and bands of the desired size were cut out and purified using a QIAQuick Gel Extraction Kit (Qiagen). DNA concentrations were measured using a Nanodrop (Thermo Scientific), and an aliquot of purified DNA was again run on 0.8\% agarose gel to check for bands of the appropriate size.\\
  
Digested inserts and shuttle vectors were ligated at 16\degree~C overnight using T4 DNA ligase (Roche). The ligation reaction was stopped by incubation at 65\degree~C for 10 minutes. An aliquot of each product was run on a 0.8\% agarose gel to check for ligation, and DNA concentrations were measured using a Qubit with the DNA High Sensitivity Assay (Life Technologies). Ligated plasmids were transformed into chemically competent \emph{E. coli} (BIOBlue from Bioline or One Shot TOP10 from Invitrogen) and plated on LA+ampicillin plates, which were incubated overnight at 37\degree~C . 24 colonies were picked for each plasmid the following day and were grown in 3 ml of LB in an orbital shaker at 37\degree~C for 24 hours. 1.5 ml of the resulting cultures were used in minipreps to isolate the plasmid DNA using the Merlin system (Ravi Iyer). 12 plasmid preparations for each construct were chosen, and an aliquot of each was digested with a restriction enzyme that was expected to cut the plasmid only once (\emph{Stu}I, \emph{Spe}I, \emph{Avr}I or \emph{EcoR}I). Digested DNA was run on a 0.8\% agarose gel to check for a band of the expected size. Of the clones that showed bands of the correct size, 4-5 for each construct were verified by Sanger sequencing.\\

One shuttle vector containing the desired insert for each construct was chosen for cloning into the final pBac{3xP3-EGFP} vector. These shuttle vectors as well as the \emph{piggyBac} vector were cut with the octo-cutter restriction enzymes \emph{Fse}I and \emph{Asc}I (NEB) as follows:  
\begin{itemize}
	\item pSLfa1180fa-Dichaete-Dam, pBac{3xP3-EGFP}: \emph{Fse}I/\emph{Asc}I
	\item pSLfa1180fa-SoxN-Dam, pBac{3xP3-EGFP}: \emph{Fse}I
	\item pSLfa1180fa-Dam, pBac{3xP3-EGFP}: \emph{Fse}I
\end{itemize}
Digestions were performed at 37\degree~C for 1.5 hours and were stopped by incubation at 65\degree~C for 20 minutes. All samples of cut pBac{3xP3-EGFP} were dephosphorylated after digestion with Antarctic phosphatase. The dephosphorylation reaction was stopped by incubation at 70\degree~C for 10 minutes. Digestion products were run out on a 0.8\% agarose gel and bands of the desired size were cut out and purified with a QIAQuick Gel Purification Kit (Qiagen). Purified DNA concentrations were measured with a Nanodrop (Thermo Scientific), and an aliquot was again run on a 0.8\% agarose gel to check for bands of the appropriate size. Digested inserts and pBac{3xP3-EGFP} vectors were ligated at 16\degree~C overnight using T4 DNA ligase (Roche). The ligation reaction was stopped by incubation at 65\degree~C for 10 minutes. An aliquot of each product was run on a 0.8\% agarose gel to check for ligation, and DNA concentrations were measured using a Qubit with the DNA High Sensitivity Assay (Life Technologies). Although some unligated pBac{3xP3-EGFP} vector was still visible in all samples, ligated vectors and inserts were also present, so transformation was attempted without further purification.\\

The ligated pBac{3xP3-EGFP}-Dichaete-Dam (pBac-Dichaete-Dam), pBac{3xP3-EGFP}-SoxN-Dam (pBac-SoxN-Dam) and pBac{3xP3}-Dam (pBac-Dam) constructs were introduced into chemically competent \emph{E. coli} (BIOBlue from Bioline or One Shot TOP10 from Invitrogen) and plated on LA+ampicillin plates, which were incubated overnight at 37\degree~C . 24 colonies were picked for each plasmid the following day and were grown in 3 ml of LB in an orbital shaker at 37\degree~C for 24 hours. 1.5 ml of the resulting cultures were used in minipreps to isolate the plasmid DNA using the Merlin system. 12 plasmid preparations for each construct were chosen, and an aliquot of each was digested with \emph{EcoR}I, which was expected to cut each plasmid in three places. Digested DNA was run on a 0.8\% agarose gel to check for three bands of the expected sizes. Additionally, three clones of pBac-Dam with the correct band sizes were chosen for verification via Sanger sequencing.\\

1-2 clones that showed the correct pattern of bands and, for pBac-Dam, had the correct insert sequence, were chosen for each construct, and the corresponding \emph{E. coli} cultures containing each plasmid were diluted in 50 ml LB and grown on an orbital shaker overnight at 37\degree~C. Plasmid DNA was purified from each culture using a Qiagen HiSpeed Plasmid Midi Kit. The concentration of purified DNA was measured using a Nanodrop (Thermo Scientific) and an aliquot was run on a 0.8\% agarose gel to check that the size was still correct. Purified plasmids were concentrated in a speedvac to a final concentration of 1 ug/ul. Plasmids were injected into embryos from each species of \emph{Drosophila} along with a plasmid containing a helper \emph{piggyBac} transposase (phsp-pBac or 1409 \emph{D. mel hsp70} hyperactive \emph{piggyBac}, supplied by Ernst Wimmer). Transformants were obtained for each construct in all species except for pBac-SoxNDam in \emph{D. pseudoobscura}; although this injection was repeated several times, no transformants were recovered.

\subsection{Isolation of DamID DNA fragments}
For each transgenic line in each species, embryos were collected after overnight lays and dechorionated in 50\% bleach. They were then rinsed in homogenization buffer (10 mM Tris-HCl pH 7.6, 60 mM NaCl, 10 mM EDTA, 0.15 mM spermine, 0.15 mM spermidine, 0.5\% Triton X-100), flash-frozen in liquid nitrogen and stored at -80\degree~C. Three biological replicates were collected from each line, with each replicate consisting of approximately 50-150 \(\mu\)l of dry embryos. To extract high-molecular weight genomic DNA, each aliquot of embryos was homogenized in a Dounce 15-ml homogenizer in 10 ml of homogenization buffer. 10 strokes were applied with pestle B, followed by 10 strokes with pestle A. The lysate was then spun for 10 minutes at 6000g. The supernatant was discarded, and the pellet was resuspended in 10 ml homogenization buffer, then spun again for 10 minutes at 6000g. The supernatant was again discarded, and the pellet was resuspended in 3 ml homogenization buffer. 300 \(\mu\)l of 20\% n-lauroyl sarcosine were added, and the samples were inverted several times to lyse the nuclei. The samples were treated with RNaseA followed by proteinase K at 37\degree~C. They were then purified by two phenol-chloroform extractions and one chloroform extraction. Genomic DNA was precipitated by adding 2X EtOH and 0.1X NaOAc, dried, and resuspended in 50-150 \(\mu\)l TE buffer, depending on the starting amount of embryos. DNA was run on a 1\% agarose gel to check for the presence of a single clean, high-molecular weight band, and the concentration was measured on a Nanodrop (Thermo Scientific).\\

Molecular biology for DamID was performed essentially as described, with some modifications \citep{vogel_detection_2007}. 30 \(\mu\)l of each gDNA sample was used in \emph{Dpn}I digestions, which were performed at 37\degree~C for two hours. Initially, distinct bands were detected after the PCR step which displayed a characteristic pattern for each species, indicating that they were likely due to the DamID primers binding non-specifically to non-digested genomic DNA. To prevent this, a size-selection step was added between the \emph{Dpn}I digestion step and the ligation step. 0.7X Agencourt AMPure XP beads (Beckman Coulter) were added to each sample to remove high-molecular weight DNA, leaving behind digested fragments. The supernatant was retained, and 1.1X Agencourt AMPure XP beads were added to recover all remaining DNA. The DNA was eluted in 30 \(\mu\)l TE buffer, then used in the ligation step. This size-selection effectively eliminated the non-specific genomic bands. For some of the non-model species of \emph{Drosophila}, faint bands were still observed at regular size intervals; it was determined that these were due to oligomerization of the DamID adapters. These were eliminated by titrating the adapters at the ligation step down to the minimum concentration that still resulted in amplification of expected products, either 1:2 or 1:4. This also eliminated the faint amplification that was sometimes visible in the no-\emph{Dpn}I control. Some adapter oligomers were still sequenced, but these could be filtered out computationally.  

\subsection{Preparation of DamID libraries for sequencing}
After PCR amplification, the DamID DNA samples were purified using a phenol-chloroform extraction followed by a chloroform extraction. The DNA was precipitated and resuspended in 50 \(\mu\)l TE buffer. They were then sonicated in order to reduce the average fragment size using a Covaris S2 sonicator with the following settings: Intensity 5, duty cycle 10\%, 200 cycles/burst, 300 seconds. After sonication, the samples were purified using a QIAquick PCR Purification kit to remove small fragments. The sample concentrations were measured using a Qubit with the DNA High Sensitivity Assay (Life Technologies), and the size distributions of DNA fragments were measured using a 2100 Bioanalyzer with the High Sensitivity DNA kit and chips (Agilent). Samples were sent to BGI Tech Solutions (HongKong) Co., Ltd., for library construction and sequencing on an Illumina MiSeq or HiSeq. Libraries were multiplexed with 2 samples per run for the MiSeq and 9-12 samples per lane for the HiSeq. MiSeq libraries were run as 150-bp single-end reads, while HiSeq libraries were run as 50-bp single-end reads. 

\section{FAIRE-seq}
\subsection{Isolation of FAIRE DNA fragments}
The timing of developmental stages for \emph{Drosophila pseudoobscura} embryos was calculated using the species-specific function from \citet{kuntz_native_2013}, with the temperature set to 25\degree~C. Adults were kept in cages at 22.5\degree~C and were given fresh grape juice-agar plates streaked with fresh yeast paste every hour for at least 2 hours before collections began. For each collection, the flies were allowed to lay for 1 hour at 22.5\degree~C, then the agar plates were removed, replaced with fresh plates, and placed at 25\degree~C for the embryos to age to the correct stage. To verify the developmental stages, an aliquot of embryos at each stage was dechorionated with 50\% bleach, devitellinized with heptane, and examined on a Zeiss Axioplan microscope with a 10x and a 20x objective. Calculated developmental times were added to the observed time that it took for embryos to reach cellularization at Stage 5, as this was considered the zero timepoint by Kuntz and Eisen (2013). The final times used for embryo staging were as follows and indicate the time that each agar plate was allowed to age after a 1-hour lay:
\begin{itemize}
	\item Stage 5: 4 hours, 35 minutes
	\item Stage 9: 6 hours
	\item Stage 10: 6 hours, 45 minutes
	\item Stage 11: 8 hours, 45 minutes
	\item Stage 14: 13 hours, 45 minutes
\end{itemize}
Staged embryos were collected, dechorionated in 50\% bleach, and fixed as described for ChIP, except that volumes were halved and fixation was for 15 minutes. Following fixation, embryos were flash-frozen in liquid nitrogen and stored at -80\degree~C. Three biological replicates were collected for each stage, with approximately 60 mg of embryos per replicate. Embryos were homogenized in a 1-ml Dounce homogenizer in 1 ml of PBT supplemented with protease inhibitors (Complete Mini Protease Inhibitor cocktail tablets, Roche). 20 strokes were applied with the loose pestle, then the lysate was allowed to rest on ice for 30 seconds, and then 20 more strokes were applied with the loose pestle. The lysate was centrifuged at 1100g for 10 minutes at 4\degree~C and the supernatant was discarded. The pellet was resuspended in 1 ml cold cell lysis buffer supplemented with protease inhibitors and further homogenized by applying 20 strokes with the tight pestle. The lysate was centrifuged at 2000g for 4 minutes at 4\degree~C to pellet the nuclei, the supernatant discarded, and the pellet resuspended in 1 ml cold nuclear lysis buffer supplemented with protease inhibitors. The mix was incubated at room temperature for 20 minutes to lyse the nuclei, split into 160-\(\mu\)l aliquots and sonicated in a Diagenode Bioruptor with the energy settings on high.\\ 

All aliquots were initially sonicated for 16 cycles of 30 seconds on, 30 seconds off. A 100-\(\mu\)l input aliquot was removed from each sample, treated with RNaseA for 30 minutes at 37\degree~C and then with proteinase K for 1 hour at 55\degree~C, and then incubated overnight at 65\degree~C to reverse crosslinks. The following day, phenol-chloroform extractions were performed on the input samples as described \citep{simon_using_2012}. The input samples were then run on a 3\% agarose gel to estimate the size distribution of DNA fragments. If the average fragment size was greater than 500 bp, the entire FAIRE sample was sonicated again for up to 4 additional cycles of 30 seconds on, 30 seconds off, and another input aliquot was taken to determine the fragment size distribution. The final fragment sizes ranged from 100 bp to 1000 bp, with the majority of fragments falling between 250 and 500 bp. Phenol-chloroform extractions were performed on the FAIRE samples as described \citep{simon_using_2012}.

\subsection{Preparation of FAIRE libraries for sequencing}
FAIRE DNA was purified using a QIAquick PCR Purification Kit (Qiagen).The concentration of each FAIRE sample was measured using a Qubit with the High Sensitivity DNA Assay (Life Technologies). The size distribution of fragments in each sample was measured using a 2100 Bioanalyzer with the High Sensitivity DNA kit and chips (Agilent). Input samples were not sequenced, but were used to gauge the percentage of the genome recovered by FAIRE \citep{simon_using_2012}. Samples with at least 10 ng of DNA present were sent to BGI Tech Solutions (HongKong) Co., Ltd., for library construction and sequencing on an Illumina HiSeq. Libraries were multiplexed with 12 samples per lane and were run as 50-bp single-end reads.

\section{Sequencing data analysis} 
\subsection{Quality control and mapping}
All high-throughput sequencing data from Illumina Hiseq and Miseq platforms was received in Fastq format. The program FastQC was used to evaluate the overall quality and attributes of each dataset (\url{http://www.bioinformatics.babraham.ac.uk/projects/fastqc/}). FastQC produces a report for each sample containing the following sections: Basic Statistics, Per Base Sequence Quality, Per Sequence Quality Scores, Per Base Sequence Content, Per Base GC Content, Per Sequence GC Content, Per Base N Content, Sequence Length Distribution, Duplicate Sequences, Overrepresented Sequences and Overrepresented Kmers. This information was used to identify potential sequencing contaminants, such as adapters. In the case of the DamID data, cutadapt was used to trim adapter sequences from both ends of reads by running the command:

\begin{lstlisting}
cutadapt -a GATCCTCGGCCGCGACC -g ^GGTCGCGGCCGAGGATC 
	-o output.fastq.gz input.fastq.gz
\end{lstlisting}

FastQC was then run again to verify that all 5' and 3' adapter sequences had been removed.\\ 

Reads were mapped against each reference genome using bowtie2 with the default settings \citep{langmead_fast_2012}. The reference genomes used were: \emph{D. melanogaster} April 2006 (UCSC dm3, BDGP) \citep{adams_genome_2000, celniker_finishing_2002}, \emph{D. simulans} April 2005 (UCSC droSim1, The Genome Institute at Washington University (WUSTL)) \citep{clark_evolution_2007}, \emph{D. yakuba} November 2005 (UCSC droYak2, The Genome Institute at Washington University (WUSTL)) \citep{clark_evolution_2007} and \emph{D. pseudoobscura} November 2004 (UCSC dp3, Baylor College of Medicine Human Genome Sequencing Center (BCM-HGSC)) \citep{clark_evolution_2007, richards_comparative_2005}. The most recent UCSC version was used for each species except \emph{D. pseudoobscura}; I decided to use the dp3 (Nov. 2004) release instead of the dp4 (Feb. 2006) release because annotation tables were not available for the dp4 release and because the unassembled chromosomes were broken into separate scaffolds, making it difficult to run certain downstream analysis tools. All reference genomes were downloaded from the UCSC Genome Browser (\url{http://hgdownload.soe.ucsc.edu/downloads.html}). Mapped reads were sorted and indexed using SAMtools \citep{li_sequence_2009}.

\subsection{ChIP-seq processing and peak calling}
All ChIP and input libraries were normalized to a total library size of 1,000,000 reads. Peaks were called on each matched pair of ChIP and input replicates independently using both MACS and Peakzilla \citep{bardet_identification_2013, zhang_model-based_2008}.

\subsection{DamID processing, peak calling and annotation}
The position of every GATC site in each genome was determined using the HOMER utility scanMotifGenomeWide.pl (\url{http://homer.salk.edu/homer/index.html}) \citep{heinz_simple_2010}. For each sample, reads were extended to the average fragment length (200 bp) using the BEDTools slop utility. The number of extended reads overlapping each GATC fragment was then calculated using the BEDTools coverage utility \citep{quinlan_bedtools:_2010}. The resulting counts for each sample were collated to form a count table for each species, consisting of one column for each fusion protein or Dam-only sample and one row for each GATC fragment in the genome. These count tables served as inputs to run DESeq2 (run in R version 3.1.0 using RStudio version 0.98), which was used to test for differential enrichment in the fusion protein samples versus the Dam-only samples in each GATC fragment \citep{love_moderated_2014}. Fragments flagged as differentially enriched (log2 fold change \textgreater 0 and adjusted p-value \textless 0.05 or \textless 0.01) were extracted, and neighboring GATC fragments with less than 100 bp separating them were merged to form binding intervals using a perl script. Binding intervals were scanned for both \emph{de novo} and known motifs using HOMER findMotifsGenome.pl \citep{heinz_simple_2010}.\\

For the \emph{D. melanogaster} data, as well as translated data from each other species, each binding interval was assigned to the closest gene using a perl script, bed2closestGene\_v2.pl, written by Bettina Fischer. Genomic feature annotations were performed using the Bioconductor package ChIPSeeker \citep{yu_chipseeker_2014}. The distances from binding intervals to TSSs were calculated and plotted using both the Bioconductor package ChIPpeakAnno \citep{zhu_chippeakanno:_2010}, which considers the distance between every interval and the closest TSS, and an unpublished suite of R scripts written by Bettina Fischer, CHIPPAVI, which considers the distance between every TSS and the surrounding intervals. All calculations of overlaps between interval datasets were performed using the BEDTools intersect utility \citep{quinlan_bedtools:_2010}.

\subsection{FAIRE-seq processing and peak calling}
Each FAIRE-seq replicate was processed separately. For each sample, reads were extended to the average fragment length (300 bp) using the BEDTools slop utility \citep{quinlan_bedtools:_2010}. MOSAiCS (run in R version 3.1.0 using RStudio version 0.98) was used to call peaks, using a one-sample analysis with 50 bp-binned mappability, GC-content and N-content scores as covariates \citep{chung_mosaics_2012}. Mappability files were generated for the \emph{D. pseudoobscura} reference genome using code available as part of the PeakSeq package (\url{http://archive.gersteinlab.org/proj/PeakSeq/Mappability_Map/Code/}) \citep{rozowsky_peakseq_2009} and supplemental code from the MOSAiCS website (\url{http://www.stat.wisc.edu/~keles/Software/mosaics/}). GC-content and N-content files were generated using the \emph{D. pseudoobscura} dp3.2bit binary file from the UCSC Genome Browser (\url{http://hgdownload.soe.ucsc.edu/downloads.html}) and supplemental code from the MOSAiCS website. After model-fitting, the Bayesian Information Criterion (BIC) and Goodness of Fit plot (GOF) for each replicate was examined, and the best-fitting model from either the one-signal-component or two-signal-component model was chosen for peak calling. Peak calling was performed at both FDR 5 and FDR 10. In order to establish a high-confidence list of peaks for each developmental stage, the FDR 10 peaks from each replicate in the same stage were intersected using BEDTools \citep{quinlan_bedtools:_2010}, and any peaks that were present in at least 2 out of the 3 replicates were kept. DiffBind (run in R version 3.1.0 using RStudio version 0.98) was used to cluster replicate samples using both peak datasets and raw reads, as well as to perform principal component analysis (PCA) of replicate read density profiles and to identify differentially enriched peaks between each developmental stage. Peaks were scanned for both de novo and known motifs using HOMER findMotifsGenome.pl \citep{heinz_simple_2010}.\\

For the analysis of FAIRE tag count in transcription factor binding intervals, peaks for the factors pipsqueak (Psq) and trithorax-like (Trl) in 0-4 hour \emph{D. pseudoobscura} embryos, and for the factors hunchback (hb), giant (gt), bicoid (bcd) and Kruppel (Kr) in blastoderm-stage \emph{D. pseudoobscura} embryos, were downloaded from GEO (accession numbers GSE25666, GSE25667 and GSE50771). Peak coordinates were translated from the dp4 assembly to the dp3 assembly of the \emph{D. pseudoobscura} genome using the UCSC LiftOver tool (\url{http://genome.ucsc.edu/cgi-bin/hgLiftOver}). For Dichaete, DamID-seq peaks called by DESeq2 with an adjusted p-value \textless 0.05 were used. Perl scripts were used to find the midpoint of each peak, then extend it 2500 bp in either direction, resulting in 5-kb intervals around each peak. These intervals were then split into 50-bp bins, and the BEDTools coverage utility was used to calculate the number of FAIRE-seq tags overlapping each bin for each biological replicate from Stage 5 \citep{quinlan_bedtools:_2010}. For each bin, the average score was taken over all peaks. FAIRE score versus position surrounding peaks was plotted in R version 3.1.0 using RStudio version 0.98.\\

Both Genscan and GeneID gene predictions for the UCSC dp3 version of the \emph{D. pseudoobscura} genome were downloaded in BED format from the UCSC Table Browser (\url{http://genome.ucsc.edu/cgi-bin/hgTables?db=dp3}) \citep{burge_prediction_1997,karolchik_ucsc_2004, karolchik_ucsc_2014,parra_geneid_2000}. The BEDTools intersect utility \citep{quinlan_bedtools:_2010} was used to classify each FAIRE peak from all stages as either exonic (falling entirely in an exon), exon boundary (partially overlapping an exon), intronic (falling entirely in an intron), gene boundary (partially overlapping a gene at either the 5' or 3' end) or intergenic (having no overlap with any gene).

\subsection{Cross-species comparison}
Both peaks and reads from non-\emph{D. melanogaster} species were translated to the \emph{D. melanogaster} UCSC dm3 reference genome using the LiftOver utility from the UCSC genome browser (\url{http://genome.ucsc.edu/cgi-bin/hgLiftOver}) \citep{bardet_computational_2011}. For \emph{D. simulans} and \emph{D. yakuba}, the minMatch parameter was set to 0.7, while for \emph{D. pseudoobscura} it was set to 0.5; for all species, multiple outputs were not permitted. To enable a quantitative comparison between DamID datasets from all species, both translated peaks and reads were analyzed with DiffBind, which was run in R v3.1.0 using RStudio v0.98 \citep{ross-innes_differential_2012}. Translated reads in BED format were converted into SAM format using a custom perl script, and then into BAM format for use with DiffBind using SAMtools \citep{li_sequence_2009}. For each analysis, the translated reads from each sample were normalized together in DiffBind using the DESeq2 normalization method. 

\subsection{Data visualization}
All visualization of sequence data was done with the Integrated Genome Browser (IGB) \citep{nicol_integrated_2009}. Sequencing coverage was visualized in WIG or BIGWIG format, while peaks were visualized in BED format. The UCSC wigToBigWig utility (\url{http://hgdownload.cse.ucsc.edu/admin/exe/}) was used to convert WIG files to the BIGWIG format for easier storage and loading.

\subsection{Code availability}
All custom perl scripts can be found at \url{www.github.com/sarahhcarl/flychip}. An overview of the DamID data processing and peak calling pipeline can be found in the wiki at:\\
\url{https://github.com/sarahhcarl/flychip/wiki/Basic-DamID-analysis-pipeline}.

\section{Molecular evolutionary analyses}
\subsection{Sequence analysis of group B Sox proteins}
All orthologous group B Sox sequences were retrieved by using BLASTX (translated BLAST: \url{http://blast.ncbi.nlm.nih.gov/Blast.cgi?PROGRAM=blastx&PAGE_TYPE=BlastSearch&LINK_LOC=blasthome}) against the genomes of \emph{D. simulans, D. yakuba} and \emph{D. pseudoobscura} with the DNA sequences of each group B \emph{Sox} gene in \emph{D. melanogaster} as queries \citep{altschul_basic_1990}. Sequences for group B Sox proteins in other species were downloaded from either NCBI (\url{http://www.ncbi.nlm.nih.gov/}) or, in the case of \emph{Aedes aegypti}, VectorBase (\url{https://www.vectorbase.org/}). Amino acid sequences were aligned and a neighbor-joining tree was constructed using ClustalW \citep{chenna_multiple_2003}. Multiple alignments were visualized using BoxShade (\url{http://ch.embnet.org/software/BOX_form.html}), and phylogenetic trees were visualized using FigTree v1.3.1\\
(\emph{http://tree.bio.ed.ac.uk/software/figtree/}).

\subsection{Multiple alignment of conserved and unique binding regions}
For the evolutionary analysis of Dichaete-Dam binding intervals, I used DiffBind to identify a set of binding intervals in \emph{D. melanogaster} that were conserved in all four species and a set that were only present in \emph{D. melanogaster} \citep{ross-innes_differential_2012}. I extracted the \emph{D. melanogaster} genome coordinates of these intervals and used the UCSC LiftOver utility to translate them to the \emph{D. simulans} droSim1, \emph{D. yakuba} droYak2 and \emph{D. pseudoobscura} dp3 reference genome assemblies. For all species, the minMatch parameter was set to 0.7. The sequences of each orthologous binding interval in each species were obtained using the fetch-UCSC sequences tool from RSAT, preserving strand information \citep{thomas-chollier_rsat_2011}. For each interval for which one unambiguous orthologous sequence could be identified in all four species, I performed a multiple alignment of the sequences using the phylogeny-aware multiple aligner PRANK \citep{loytynoja_algorithm_2005,loytynoja_phylogeny-aware_2008}. In each case, I estimated the guide tree from the data directly, resulting in a calculation of the substitution rate on each branch. I decided to estimate the guide tree from the data rather than using an independent estimate of branch lengths because selection is expected to act differently on different classes of DNA; therefore, branch lengths determined using coding sequences or averaging over the genome might over- or under-estimate the expected differences in regulatory sequences between species. Note, however, that the branch lengths estimated by prank are quite close to those determined from 1000 random 10-kb noncoding regions in \emph{D. melanogaster, D. simulans, D. yakuba} and \emph{D. erecta} by \citet{moses_large-scale_2006}. I calculated the percentage of perfectly conserved nucleotides in each interval from these multiple alignments using a custom Perl script available at \url{www.github.com/sarahhcarl/Flychip/DamID_analysis}.

\subsection{Predicting transcription factor binding sites}
Two different strategies were used to predict Sox binding sites within the DamID binding intervals. All binding interval sequences were scanned independently for matches to Sox motifs using FIMO \citep{grant_fimo:_2011}, and subsets of aligned sequences were scanned using the RSAT tool matrix-scan \citep{sand_analyzing_2008,turatsinze_using_2008}. In each case, I first downloaded the positional weight matrix (PWMs) representing the top-scoring \emph{de novo} Sox motif identified via HOMER in each DamID binding interval dataset \citep{heinz_simple_2010}. I then used FIMO to search for matches to each of these PWMs in all DamID binding datasets, using the original sequences from each species’ genome. The resulting hits were used to calculate the average number of motifs per binding interval overall, as well as in binding intervals that are conserved in all species versus those that are unique to one.\\

I used the same PWMs to scan the multiple alignments of both 4-way conserved and unique Dichaete-Dam binding intervals for potential Sox binding sites using matrix-scan. I chose matrix-scan for this analysis because, unlike FIMO, it can accept multiple alignments directly as input, greatly facilitating the assignment of positional orthology to putative binding sites. Matrix-scan was run using the pre-compiled \emph{Drosophila} background file provided by RSAT as the background for scanning and with the cutoff for reporting matches set to a PWM weight-score of \textgreater=4 and a p-value of \textless 0.0001. If a binding site was identified at the same aligned position in an orthologous binding interval in more than one species, it was considered to be an orthologous site between those species. Sites that partially overlapped in position between orthologous enhancers were not considered to be orthologous; however, these were not common. Sites identified as matching multiple, overlapping PWMs in the same species and dataset were considered as separate hits. 

\subsection{Tests of conservation}
The percentages of conserved nucleotides present in binding intervals and motifs were calculated using custom perl scripts available at \url{https://github.com/sarahhcarl/Flychip}. Randomly shuffled control motifs were generated using the RSAT tool permute-matrix \citep{thomas-chollier_rsat_2011}. All statistical tests were performed in R v3.1.0 using RStudio v0.98.
