\appendix
\chapter{Glossary}
\begin{itemize}
  \item \textbf{Black box machine}: Complex machine composed of a large number of physical parts carrying a certain number of internal functions and acting together in an organised fashion.
  \item \textbf{Description logics}: Family of formal languages used to represent the knowledge of a domain of interest.
  \item \textbf{Disease}: Impairement of a normal biological condition.
  \item \textbf{Drug}: Large or small molecule producing a pharmacological effect when administred in an organism. In this document, this word is interchangeably used with the word \emph{compound}.
  \item \textbf{Drug repositioning}: Identification of new therapeutic indications for known drugs. Also referred as \emph{drug repurposing}, \emph{re-profiling}, \emph{therapeutic switching} and \emph{drug re-tasking}.
  \item \textbf{OWL2 EL}: Profile of the Web Ontology Language implementing the axioms described in the description logic \dl{EL\textsuperscript{++}} family. 
  \item \textbf{Gene expression experiment}: Quantitative or qualitative characterisation of the number of messenger RNA molecules present in a biological system at a given time and location. This number reflects the expression or activity of genes of interest.
  \item \textbf{Indication}: Therapeutic usage of a drug, regulated by an authority.
  \item \textbf{Knowledge base}: Set of description logics axioms, entities and expressions.
  \item \textbf{Machine}: Assembly of parts functioning to meet a particular goal.
  \item \textbf{Mechanism of action}: Biochemical interaction that gives rise to the pharmacological effect of a drug.
  \item \textbf{Mode of action}: Broad activity of the molecule on an organism.
  \item \textbf{Ontology}: Specification of conceptualisation, formal representation of the knowledge of a domain of interest.
  \item \textbf{Off-label use}: Prescription of an approuved drug for a non-regulated indication.
  \item \textbf{Off-target}: Said to the secondary molecular entities affected by a drug. Off-targets are generally not desirable as they can produce side-effects.
  \item \textbf{Organism}: Assembly of molecules functioning as a stable whole.
  \item \textbf{Phenotype}: Set of characteristics or traits attributed to an organism.
  \item \textbf{Polypharmacology}: Potential of a drug to produce multiple pharmacological effects.
  \item \textbf{Reasoning}: Classification and consistency checking of a knowledge base. Also referred as \emph{classification}.
  \item \textbf{Semantic similarity}: Value reflecting the closeness between two concepts or entities.
  \item \textbf{Similar property principle}: Molecules with similar structures produce similar pharmacological effects.
\end{itemize}

\chapter{Abbreviations}
\begin{itemize}
  \item \textbf{CMap}: Connectivity Map.
  \item \textbf{DLs}: Description logics.
  \item \textbf{GO}: Gene Ontology.
  \item \textbf{GWAS}: Genome wide association study.
  \item \textbf{MoA}: Mode of action.
  \item \textbf{RDF}: Resource Description Framework.
  \item \textbf{SNP}: Single nucleotide polymorphism.
  \item \textbf{OBO}: Open biological and biomedical ontologies.
  \item \textbf{OWL}: Web Ontology Language.
\end{itemize}

\chapter{Peer-reviewed publications}
\begin{itemize}
  \item \textbf{balbalab}: babalabalb.
\end{itemize}

\chapter{Side-projects}
\begin{itemize}
  \item \textbf{balbalab}: babalabalb.
\end{itemize}
